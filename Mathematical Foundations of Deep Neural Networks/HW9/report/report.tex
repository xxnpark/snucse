\documentclass[10pt]{article}
%\usepackage[left=2.3cm,right=2.3cm,top=2.5cm,bottom=3cm,a4paper]{geometry}
\usepackage{fullpage}
\usepackage{setspace}
\setstretch{1.3}
\usepackage{amsmath,amssymb,amsthm,physics,units,mathtools,bm}
\usepackage{mdwlist}
\usepackage{paralist}
\setlength\parindent{0pt}
\usepackage{float}
\usepackage{xcolor}
\usepackage{algorithm,algpseudocode}
\usepackage{listings}
\usepackage[colorlinks=true]{hyperref}
\setcounter{MaxMatrixCols}{50}
\NewDocumentCommand{\code}{v}{%
\texttt{#1}%
}

\definecolor{dkgreen}{rgb}{0,0.6,0}
\definecolor{gray}{rgb}{0.5,0.5,0.5}
\definecolor{lgray}{RGB}{240,239,239}
\definecolor{mauve}{rgb}{0.58,0,0.82}

\lstset{
    % frame=tb,
    aboveskip=3mm,
    belowskip=3mm,
    showstringspaces=false,
    columns=flexible,
    basicstyle=\ttfamily,
    numbers=none,
    backgroundcolor=\color{lgray},
    numberstyle=\tiny\color{gray},
    keywordstyle=\color{blue},
    commentstyle=\color{dkgreen},
    stringstyle=\color{mauve},                
    keepspaces=true,
    breaklines=true,
    breakatwhitespace=true,
    tabsize=3
}

\begin{document}
\begin{center}
    {\LARGE MathDNN Homework 9} \\
\end{center}
\begin{flushright}
    Department of Computer Science and Engineering \\
    2021-16988 Jaewan Park
\end{flushright}

\section*{Problem 3}
Consider $\Omega$ and $\Omega^\complement$ as ordered and sorted sets. 
Now define $f$ and $g$ as $f(x) = i|_{x\text{ is the }i\text{th element of }\Omega}$ and $g(y) = j|_{y\text{ is the }j\text{th element of }\Omega^\complement}$ for $x \in \Omega$ and $y \in \Omega^\complement$ each.
We can calculate the Jacobian matrix between the layers in the form of
\begin{equation*}
    \frac{\partial z}{\partial x} = \qty{\frac{\partial z_i}{\partial x_j}}_{i, j}, \;\;\;\;
    \frac{\partial z_i}{\partial x_j} = \begin{cases}
        1 & \Big(i \in \Omega, \; i = j\Big) \\
        \frac{\partial \qty[s_\theta(x_\Omega)]_{g(i)}}{\partial x_j}e^{\qty[s_\theta(x_\Omega)]_{g(i)}}x_i + \frac{\partial \qty[t_\theta(x_\Omega)]_{g(i)}}{\partial x_j} & \qty(i \in \Omega^\complement, \; j \in \Omega) \\
        e^{\qty[s_\theta(x_\Omega)]_{g(i)}} \qty(= e^{\qty[s_\theta(x_\Omega)]_{g(j)}}) & \qty(i \in \Omega^\complement, \; j \in \Omega^\complement, \; i = j) \\
        0 & \Big(\text{otherwise}\Big)
    \end{cases}.
\end{equation*}
Selecting $\sigma$ such that $\sigma^{-1}(i) = \begin{cases}
    f^{-1}(i) & (i \leq |\Omega|) \\
    g^{-1}\qty(i - |\Omega|) & (i > |\Omega|)
\end{cases}$ gives
\begin{align*}
    P_\sigma\frac{\partial z}{\partial x}P_{\sigma^{-1}} &= \begin{bmatrix}
        {\partial z_{\sigma^{-1}(1)}}/{\partial x_{\sigma^{-1}(1)}} & {\partial z_{\sigma^{-1}(1)}}/{\partial x_{\sigma^{-1}(2)}} & \cdots & {\partial z_{\sigma^{-1}(1)}}/{\partial x_{\sigma^{-1}(n)}} \\
        {\partial z_{\sigma^{-1}(2)}}/{\partial x_{\sigma^{-1}(1)}} & {\partial z_{\sigma^{-1}(2)}}/{\partial x_{\sigma^{-1}(2)}} & \cdots & {\partial z_{\sigma^{-1}(2)}}/{\partial x_{\sigma^{-1}(n)}} \\
        \vdots & \vdots & \ddots & \vdots \\
        {\partial z_{\sigma^{-1}(n)}}/{\partial x_{\sigma^{-1}(1)}} & {\partial z_{\sigma^{-1}(n)}}/{\partial x_{\sigma^{-1}(2)}} & \cdots & {\partial z_{\sigma^{-1}(n)}}/{\partial x_{\sigma^{-1}(n)}} \\
    \end{bmatrix} \\
    &= \begin{bmatrix}
        I & 0 \\
        * & \mathrm{diag}\qty(e^{s_\theta\qty(x_\Omega)})
    \end{bmatrix}.
\end{align*}
Therefore $\dfrac{\partial z}{\partial x}$ can be decomposed in the form of 
\begin{equation*}
    \frac{\partial z}{\partial x} = P_{\sigma^{-1}}\begin{bmatrix}
        I & 0 \\
        * & \mathrm{diag}\qty(e^{s_\theta\qty(x_\Omega)})
    \end{bmatrix}P_\sigma,
\end{equation*}
and we can calculate the determinant as
\begin{align*}
    \log\qty|\frac{\partial z}{\partial x}| &= \log\begin{vmatrix}
        I & 0 \\
        * & \mathrm{diag}\qty(e^{s_\theta\qty(x_\Omega)})
    \end{vmatrix} \\
    &= \log\prod_{i \in \Omega^\complement}e^{\qty[s_\theta\qty(x_\Omega)]_{g(i)}} = \sum_{i \in \Omega^\complement}\qty[s_\theta\qty(x_\Omega)]_{g(i)} \\
    &= \mathbf{1}_{n-|\Omega|}^\intercal s_\theta\qty(x_\Omega).
\end{align*}

\section*{Problem 4}
\begin{enumerate}[(a)]
    \item Since $-\log$ is a convex function, we can apply Jensen's inequality to $-\log$, which gives
    \begin{align*}
        D_{\mathrm{KL}}(X||Y) &= \int_{\mathbb{R}^d}f(x)\log\qty(\frac{f(x)}{g(x)})dx = \mathbf{E}\qty[\log\qty(\frac{f(X)}{g(X)})] = \mathbf{E}\qty[-\log\qty(\frac{g(X)}{f(X)})] \\
        &\geq -\log\qty(\mathbf{E}\qty[\frac{g(X)}{f(X)}]) = -\log\qty(\int_{\mathbb{R}^d}f(x)\cdot\frac{g(x)}{f(x)}dx) = -\log{1} = 0.
    \end{align*}
    \item Since $X_1, \cdots, X_d$ and $Y_1, \cdots, Y_d$ are each independent, when $f_1, \cdots, f_d$ and $g_1, \cdots, g_d$ are PDFs for $X_1, \cdots, X_d$ and $Y_1, \cdots, Y_d$ each,
    we can say
    \begin{equation*}
        f(x) = f_1\qty(x_1) \cdots f_d\qty(x_d), \;\; g(y) = g_1\qty(y_1) \cdots g_d\qty(y_d)
    \end{equation*}
    for any $x = \qty(x_1, \cdots, x_d)$ and $y = \qty(y_1, \cdots, y_d)$.
    Therefore
    \begin{align*}
        D_{\mathrm{KL}}(X||Y) &= \mathbf{E}\qty[-\log\qty(\frac{g(X)}{f(X)})] = \mathbf{E}\qty[-\log\qty(\frac{g_1\qty(X_1)}{f_1\qty(X_1)})] + \cdots + \mathbf{E}\qty[-\log\qty(\frac{g_d\qty(X_d)}{f_d\qty(X_d)})] \\
        &= D_{\mathrm{KL}}(X_1||Y_1) + \cdots + D_{\mathrm{KL}}(X_d||Y_d).
    \end{align*}
\end{enumerate}

\section*{Problem 5}
The PDF of a multivariate Gaussian random variable $X \sim \mathcal{N}\qty(\mu, \Sigma)$ with dimension $d$ is given by 
\begin{equation*}
    p_X(x) = \frac{1}{\sqrt{(2\pi)^{d}\det\Sigma}}\exp\qty(-\frac{1}{2}\qty(x-\mu)^\intercal\Sigma^{-1}\qty(x-\mu)).
\end{equation*}
Let $X_0$, $X_1$ random variables that follow $\mathcal{N}\qty(\mu_0, \Sigma_0)$, $\mathcal{N}\qty(\mu_1, \Sigma_1)$ each.
Also let their PDFs $f_0$, $f_1$.
Then
\begin{align*}
    D_{\mathrm{KL}}\qty(\mathcal{N}\qty(\mu_0, \Sigma_0) || \mathcal{N}\qty(\mu_1, \Sigma_1)) \\
    &\hspace{-3cm}= \mathbf{E}\qty[-\log\qty(\frac{f_1(X_0)}{f_0(X_0)})] = \mathbf{E}\biggl[\log{f_0(X_0)} - \log{f_1(X_0)}\biggr] \\
    &\hspace{-3cm}= \mathbf{E}\qty[\frac{1}{2}\log\frac{\det\Sigma_1}{\det\Sigma_0} - \frac{1}{2}\qty(X_0-\mu_0)^\intercal\Sigma_0^{-1}\qty(X_0-\mu_0) + \frac{1}{2}\qty(X_0-\mu_1)^\intercal\Sigma_1^{-1}\qty(X_0-\mu_1)] \\
    &\hspace{-3cm}= \frac{1}{2}\log\frac{\det\Sigma_1}{\det\Sigma_0} - \frac{1}{2}\mathbf{E}\qty[\mathrm{tr}\qty(\qty(X_0-\mu_0)^\intercal\Sigma_0^{-1}\qty(X_0-\mu_0))] + \frac{1}{2}\mathbf{E}\qty[\qty(X_0-\mu_1)^\intercal\Sigma_1^{-1}\qty(X_0-\mu_1)] \\
    &\hspace{-3cm}= \frac{1}{2}\log\frac{\det\Sigma_1}{\det\Sigma_0} - \frac{1}{2}\mathbf{E}\qty[\mathrm{tr}\qty(\qty(X_0-\mu_0)\qty(X_0-\mu_0)^\intercal\Sigma_0^{-1})] + \frac{1}{2}\qty(\qty(\mu_0-\mu_1)^\intercal\Sigma_1^{-1}\qty(\mu_0-\mu_1) + \mathrm{tr}\qty(\Sigma_1^{-1}\Sigma_0)) \\
    &\hspace{-3cm}= \frac{1}{2}\log\frac{\det\Sigma_1}{\det\Sigma_0} - \frac{1}{2}\mathrm{tr}\qty(\mathbf{E}\qty[\qty(X_0-\mu_0)\qty(X_0-\mu_0)^\intercal]\Sigma_0^{-1}) + \frac{1}{2}\qty(\qty(\mu_1-\mu_0)^\intercal\Sigma_1^{-1}\qty(\mu_1-\mu_0) + \mathrm{tr}\qty(\Sigma_1^{-1}\Sigma_0)) \\
    &\hspace{-3cm}= \frac{1}{2}\log\frac{\det\Sigma_1}{\det\Sigma_0} - \frac{1}{2}\mathrm{tr}\qty(\Sigma_0\Sigma_0^{-1}) + \frac{1}{2}\qty(\qty(\mu_1-\mu_0)^\intercal\Sigma_1^{-1}\qty(\mu_1-\mu_0) + \mathrm{tr}\qty(\Sigma_1^{-1}\Sigma_0)) \\
    &\hspace{-3cm}= \frac{1}{2}\qty(\mathrm{tr}\qty(\Sigma_1^{-1}\Sigma_0) + \qty(\mu_1-\mu_0)^\intercal\Sigma_1^{-1}\qty(\mu_1-\mu_0) - d + \log\qty(\frac{\det\Sigma_1}{\det\Sigma_0})).
\end{align*}

\section*{Problem 6}
For each $\theta$, let $\phi_\theta \in \Phi$ the value of $\phi$ that makes $h(\theta, \phi) = 0$.
Then we obtain
\begin{align*}
    \sup_{\theta, \phi}g(\theta, \phi) &= \sup_{\theta}\qty(\sup_{\phi}g(\theta, \phi)) \\
    &= \sup_{\theta}\qty(\sup_{\phi}\Big(f(\theta) - h(\theta, \phi)\Big)) = \sup_{\theta}\qty(f(\theta) - \inf_{\phi}h(\theta, \phi)) \\
    &= \sup_{\theta}f(\theta)
\end{align*}
since $\inf_{\phi}h(\theta, \phi) = 0$, more precisely $\min_{\phi}h(\theta, \phi) = 0$ when $\phi = \phi_\theta$.
Therefore we can conclude that
$$\mathrm{argmax}\,f = \qty{\theta \,|\, (\theta, \phi) \in \mathrm{argmax}\,g}$$
and the two given optimization problems are equivalent.

\end{document}