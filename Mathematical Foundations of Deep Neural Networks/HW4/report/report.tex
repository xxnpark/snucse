\documentclass[10pt]{article}
%\usepackage[left=2.3cm,right=2.3cm,top=2.5cm,bottom=3cm,a4paper]{geometry}
\usepackage{fullpage}
\usepackage{setspace}
\setstretch{1.3}
\usepackage{amsmath,amssymb,amsthm,physics,units,mathtools}
\usepackage[shortlabels]{enumitem}
\setlength\parindent{0pt}
\usepackage{float}
\usepackage{multicol}
\usepackage{algorithm,algpseudocode}
\usepackage[shortlabels]{enumitem}
\usepackage{hyperref}
\hypersetup{
    colorlinks=true,
    linkcolor=black,
    filecolor=black,      
    urlcolor=black,
    pdftitle={Overleaf Example},
    pdfpagemode=FullScreen,
}

\begin{document}
\begin{center}
    {\LARGE MathDNN Homework 4} \\
\end{center}
\begin{flushright}
    Department of Computer Science and Engineering \\
    2021-16988 Jaewan Park
\end{flushright}

\section*{Problem 1}
The parameters for convolution are $C_{\mathrm{in}} = 1$, $C_{\mathrm{out}} = 2$, $F = 3$, $S = 1$, $P = 1$.
Then the relationship between $X$ and $Y$ becomes
$$Y_{l, i, j} = \sum_{\alpha = 1}^{F}\sum_{\beta = 1}^{F}w_{l, \alpha, \beta}X_{S(i - 1) - P + \alpha, \: S(j - 1) - P + \beta} + b_{l}$$
Therefore,
$$Y_{1, i, j} = \sum_{\alpha = 1}^{3}\sum_{\beta = 1}^{3}w_{1, \alpha, \beta}X_{i + \alpha - 2, j + \beta - 2} + b_{1} = X_{i+1, j} - X_{i, j}$$
$$Y_{2, i, j} = \sum_{\alpha = 1}^{3}\sum_{\beta = 1}^{3}w_{2, \alpha, \beta}X_{i + \alpha - 2, j + \beta - 2} + b_{2} = X_{i, j+1} - X_{i, j}$$
$$\therefore w_{1, \alpha, \beta} = \begin{cases}
    1  & (\alpha = 3, \; \beta = 2) \\
    -1 & (\alpha = 2, \; \beta = 2) \\
    0  & (\textrm{otherwise})
\end{cases}, \;\; w_{2, \alpha, \beta} = \begin{cases}
    1  & (\alpha = 2, \; \beta = 3) \\
    -1 & (\alpha = 2, \; \beta = 2) \\
    0  & (\textrm{otherwise})
\end{cases}, \;\; b = 0.$$

\section*{Problem 2}
A common 2D convolution has the following relationship between $X$ and $Y$. (Assume the batch size is 1.)
$$Y_{l, i, j} = \sum_{\gamma = 1}^{C_{\mathrm{in}}}\sum_{\alpha = 1}^{f_1}\sum_{\beta = 1}^{f_2}w_{l, \gamma, \alpha, \beta}X_{\gamma, S(i-1) + \alpha, S(j-1) + \beta} + b_{l}$$
In the case of the given average pooling operation, $C_{\mathrm{in}} = C_{\mathrm{out}} = C$, $f_1 = f_2 = S = k$, $b=0$, $P=0$, and $$w_{l,\gamma, \alpha, \beta} = \begin{cases}
    \dfrac{1}{k^2} & (\gamma = l) \\
    0 & (\mathrm{otherwise})
\end{cases}.$$
Substituting these to the above equation gives
\begin{align*}
    Y_{l, i, j} &= \sum_{\gamma = 1}^{C}\sum_{\alpha = 1}^{k}\sum_{\beta = 1}^{k}w_{l, \gamma, \alpha, \beta}X_{\gamma, k(i-1) + \alpha, k(j-1) + \beta} \\
    &= \frac{1}{k^2}\sum_{\alpha = 1}^{k}\sum_{\beta = 1}^{k}X_{l, k(i-1) + \alpha, k(j-1) + \beta}
\end{align*}
which equals to the given relationship where indices $\alpha, \beta, l$ correspond to $a, b, c$.
Also, 
$$W_{\mathrm{out}} = \left\lfloor{\frac{W_\mathrm{in} - f_1 + 2P}{S} + 1}\right\rfloor = \left\lfloor{\frac{m - k + 0}{k} + 1}\right\rfloor = \frac{m}{k}$$
$$H_{\mathrm{out}} = \left\lfloor{\frac{H_\mathrm{in} - f_2 + 2P}{S} + 1}\right\rfloor = \left\lfloor{\frac{m - k + 0}{k} + 1}\right\rfloor = \frac{m}{k}$$
so the input and output dimensions match with the given conditions.
In this way, average pooling can be represented as a convolution.

\section*{Problem 3}
Consider an unbiased convolution and set parameters $C_{\mathrm{in}} = 3$, $C_{\mathrm{out}} = 1$, $F = 1$, $S = 1$, $P = 0$. Then substituting these parameters gives
\begin{align*}
    Y_{i, j} &= \sum_{\gamma = 1}^{C_{\mathrm{in}}}\sum_{\alpha = 1}^{F}\sum_{\beta = 1}^{F}w_{\gamma, \alpha, \beta}X_{\gamma, S(i-1) + \alpha, S(j-1) + \beta} + b_{l} \\
    &= w_{1, 1, 1}X_{1, i, j} + w_{2, 1, 1}X_{2, i, j} + w_{3, 1, 1}X_{3, i, j}
\end{align*}
Therefore $w_{1, 1, 1} = 0.299$, $w_{2, 1, 1} = 0.587$, $w_{3, 1, 1} = 0.114$.

\section*{Problem 4}
Two functions $\sigma$ and $\rho$ are commute as the following.
\begin{align*}
    \qty[\sigma(\rho(X))]_{i, j} &= \sigma\qty(\qty[\rho(X)]_{i, j}) \\
    &= \sigma\qty(\max_{1 \leq p \leq \frac{m}{k}, \;  1 \leq q \leq \frac{n}{l}}{X_{\frac{m}{k}(i-1) + p, \frac{n}{l}(j-1) + q}}) \\
    &= \max_{1 \leq p \leq \frac{m}{k}, \;  1 \leq q \leq \frac{n}{l}}\sigma\qty(X_{\frac{m}{k}(i-1) + p, \frac{n}{l}(j-1) + q}) \;\; (\because \sigma\text{ is a nondecreasing function}) \\
    &= \max_{1 \leq p \leq \frac{m}{k}, \;  1 \leq q \leq \frac{n}{l}}\qty[\sigma\qty(X)]_{\frac{m}{k}(i-1) + p, \frac{n}{l}(j-1) + q} \\
    &= \qty[\rho(\sigma(X))]_{i, j}
\end{align*}

\section*{Problem 6}
\textit{Notation} For a matrix or vector $X$, the notation $[X]_{i, j}$ or $[X]$ refers to the element of $X$ at that index, and $\qty{f(i, j)}_{i, j}$ refers to a matrix of which element at $(i, j)$ is $f(i, j)$.
\begin{enumerate}[leftmargin=*, label=(\alph*)]
    \item Since $y_L = A_Ly_{L-1} + b_L$, it is trivial that 
    $$\frac{\partial y_L}{\partial b_L} = 1, \;\; \frac{\partial y_L}{\partial y_{L-1}} = A_L.$$
    For $y_\ell = \sigma\qty(A_\ell y_{\ell-1} + b_\ell)$, we can obtain the following.
    \begin{align*}
        \frac{\partial y_\ell}{\partial b_\ell} &= \frac{\partial}{\partial b_\ell}\sigma\qty(A_\ell y_{\ell-1} + b_\ell)\\
        &= \qty{\frac{\partial}{\partial\qty[b_\ell]_j}\qty[\sigma\qty(A_\ell y_{\ell-1} + b_\ell)]_i}_{i, j} = \qty{\frac{\partial}{\partial\qty[b_\ell]_j}\sigma\qty(\qty[A_\ell y_{\ell-1} + b_\ell]_i)}_{i, j} \\
        &= \qty{\frac{\partial}{\partial\qty[b_\ell]_j}\sigma\qty(\qty[A_\ell y_{\ell-1}]_i + \qty[b_\ell]_i)}_{i, j} \\
        &= \qty{\sigma'\qty(\qty[A_\ell y_{\ell-1}]_i + \qty[b_\ell]_i)\cdot\frac{\partial}{\partial\qty[b_\ell]_j}\qty(\qty[A_\ell y_{\ell-1}]_i + \qty[b_\ell]_i)}_{i, j} \\
        &= \qty{\sigma'\qty(\qty[A_\ell y_{\ell-1}]_i + \qty[b_\ell]_i)\cdot\delta_{ij}}_{i, j} \;\; (\delta_{ij} \text{: Kronecker Delta})\\
        &= \text{diag}\:\Big(\sigma'\qty(A_\ell y_{\ell-1} + b_\ell)\Big)
    \end{align*}
    \begin{align*}
        \frac{\partial y_\ell}{\partial y_{\ell-1}} &= \frac{\partial}{\partial y_{\ell-1}}\sigma\qty(A_\ell y_{\ell-1} + b_\ell) \\
        &= \qty{\sigma'\qty(\qty[A_\ell y_{\ell-1}]_i + \qty[b_\ell]_i)\cdot\frac{\partial}{\partial\qty[y_{\ell-1}]_j}\qty(\qty[A_\ell y_{\ell-1}]_i + \qty[b_\ell]_i)}_{i, j} \\
        &= \qty{\sigma'\qty(\qty[A_\ell y_{\ell-1}]_i + \qty[b_\ell]_i)\cdot\frac{\partial}{\partial\qty[y_{\ell-1}]_j}\qty(\sum_{k=1}^{n_{\ell-1}}\qty[A_\ell]_{i,k}\qty[y_{\ell-1}]_k + \qty[b_\ell]_i)}_{i, j} \\
        &= \qty{\sigma'\qty(\qty[A_\ell y_{\ell-1}]_i + \qty[b_\ell]_i)\cdot\qty[A_\ell]_{i, j}}_{i, j} \\
        &= \text{diag}\:\Big(\sigma'\qty(A_\ell y_{\ell-1} + b_\ell)\Big) A_\ell
    \end{align*}
    \item Since $y_L = A_Ly_{L-1} + b_L$, 
    \begin{align*}
        \qty[\frac{\partial y_L}{\partial A_L}]_{1, j} &= \frac{\partial y_L}{\partial \qty[A_L]_{1, j}} \\
        &= \frac{\partial}{\partial \qty[A_L]_{1, j}}\qty(A_L y_{L-1} + b_L) = \frac{\partial}{\partial \qty[A_L]_{1, j}}\qty(\sum_{k=1}^{n_{\ell-1}}\qty[A_L]_{1,k}\qty[y_{L-1}]_k) \\
        &= \qty[y_{L-1}]_j
    \end{align*}
    so $\dfrac{\partial y_L}{\partial A_L} = y_{L-1}^\intercal$. For $\ell = 1, \cdots, L-1$, we can obtain the following.
    \begin{align*}
        \qty[\frac{\partial y_L}{\partial A_\ell}]_{i, j} &= \frac{\partial y_L}{\partial \qty[A_\ell]_{i, j}} = \frac{\partial y_L}{\partial y_\ell}\frac{\partial y_\ell}{\partial \qty[A_\ell]_{i, j}} \\
        &= \frac{\partial y_L}{\partial y_\ell}\qty{\frac{\partial\qty[y_\ell]_k}{\partial \qty[A_\ell]_{i, j}}}_{k} \\
        &= \frac{\partial y_L}{\partial y_\ell}\qty{\frac{\partial}{\partial \qty[A_\ell]_{i, j}}\sigma\qty(\qty[A_\ell y_{\ell-1}]_k + \qty[b_\ell]_k)}_{k} \\
        &= \frac{\partial y_L}{\partial y_\ell}\qty{\sigma'\qty(\qty[A_\ell y_{\ell-1}]_k + \qty[b_\ell]_k)\qty(\frac{\partial}{\partial \qty[A_\ell]_{i, j}}\qty(\qty[A_\ell y_{\ell-1}]_k + \qty[b_\ell]_k))}_{k} \\
        &= \frac{\partial y_L}{\partial y_\ell}\begin{bmatrix}
            0 \\ \vdots \\ {\sigma'\qty(\qty[A_\ell y_{\ell-1}]_i + \qty[b_\ell]_i)\qty[y_{\ell - 1}]_j} \\ \vdots \\ 0
        \end{bmatrix} \;\; (\text{All elements except the } i \text{-th element are }0.)\\
        &= \qty[\frac{\partial y_L}{\partial y_\ell}]_{i}\sigma'\qty(\qty[A_\ell y_{\ell-1}]_i + \qty[b_\ell]_i)\qty[y_{\ell - 1}]_j \\
        &= \qty[\sigma'\qty(A_\ell y_{\ell-1} + b_\ell)]_i\qty[\frac{\partial y_L}{\partial y_\ell}]_{i}\qty[y_{\ell - 1}]_j
    \end{align*}
    Therefore $\dfrac{\partial y_L}{\partial A_\ell} = \text{diag}\:\qty(\sigma'\qty(A_\ell y_{\ell-1} + b_\ell))\qty(\dfrac{\partial y_L}{\partial y_\ell})^\intercal y_{\ell-1}^\intercal$.

\end{enumerate}

\end{document}