\documentclass[10pt]{article}
%\usepackage[left=2.3cm,right=2.3cm,top=2.5cm,bottom=3cm,a4paper]{geometry}
\usepackage{fullpage}
\usepackage{setspace}
\setstretch{1.3}
\usepackage{amsmath,amssymb,amsthm,physics,units}
\usepackage[shortlabels]{enumitem}
\setlength\parindent{0pt}
\usepackage{float}
\usepackage{algorithm,algpseudocode}
\usepackage[shortlabels]{enumitem}
\usepackage{hyperref}
\hypersetup{
    colorlinks=true,
    linkcolor=black,
    filecolor=black,      
    urlcolor=black,
    pdftitle={Overleaf Example},
    pdfpagemode=FullScreen,
}

\begin{document}
\begin{center}
    {\LARGE MathDNN Homework 2} \\
\end{center}
\begin{flushright}
    Department of Computer Science and Engineering \\
    2021-16988 Jaewan Park
\end{flushright}

\section*{Problem 4}
Let $\varphi(x) = -\log(x)$, then $\varphi''(x) = 1/x^2$ is nonnegative for all $x \in \mathbb{R}^+$. Therefore $-\log(x)$ is a convex function.
Since $\varphi(x)$ is defined over $\mathbb{R}^+$ which is a convex set (it is obvious that $\forall x_1, x_2 \in \mathbb{R}^+$, $\eta x_1 + (1-\eta)x_2 \;(> 0) \in \mathbb{R}^+$), we can apply Jensen's inequality to $-\log(x)$.
Therefore we obtain
\begin{align*}
    D_{KL}(p||q) &= \sum_{i=1}^{N}p_i\log\qty(\frac{p_i}{q_i}) = \sum_{i=1}^{N}p_i\qty(-\log(\frac{q_i}{p_i})) = \sum_{i=1}^{N}p_i\varphi(\frac{q_i}{p_i})\\
    &= \mathbb{E}\qty[\varphi\qty(\frac{q_i}{p_i})] \\
    &\geq \varphi\qty(\mathbb{E}\qty[\frac{q_i}{p_i}]) \\
    &= -\log\qty(\sum_{i=1}^{N}p_i\frac{q_i}{p_i}) = -\log\qty(\sum_{i=1}^{N}q_i) = -\log 1 \\
    &= 0
\end{align*}
when we let ${q_i}/{p_i}$ a random variable with probability mass $p_i$.
In cases where $p_i = 0$ or $q_i = 0$, the value of $p_i\log\qty(p_i/q_i)$ is considered either $0$ or $\infty$, so the sum maintains nonnegative.

\section*{Problem 5}
Let $\varphi(x) = -\log(x)$, then $\varphi''(x) = 1/x^2$ is positive for all $x \in \mathbb{R}^+$. Therefore $-\log(x)$ is a strictly convex function.
As described in \textbf{Problem 4}, $\mathbb{R}^+$ is a convex set, and if we let $X = {q_i}/{p_i}$ a random variable $X$ is non constant.
Therefore we can apply the strict Jensen's inequality to $-\log(x)$. 
Therefore we obtain
\begin{align*}
    D_{KL}(p||q) &= \sum_{i=1}^{N}p_i\log\qty(\frac{p_i}{q_i}) = \sum_{i=1}^{N}p_i\qty(-\log(\frac{q_i}{p_i})) = \sum_{i=1}^{N}p_i\varphi(\frac{q_i}{p_i})\\
    &= \mathbb{E}\qty[\varphi\qty(\frac{q_i}{p_i})] \\
    &> \varphi\qty(\mathbb{E}\qty[\frac{q_i}{p_i}]) \\
    &= -\log\qty(\sum_{i=1}^{N}p_i\frac{q_i}{p_i}) = -\log\qty(\sum_{i=1}^{N}q_i) = -\log 1 \\
    &= 0.
\end{align*}
For cases where $p_i = 0$ or $q_i = 0$ the inequality stil remains. If $p_i > 0$ and $q_i = 0$, $p_i\log\qty(\dfrac{p_i}{0}) = \infty$, so the sum still maintains positive.
If $p_i = 0$ and $q_i = 0$,$p_i\log\qty(\dfrac{p_i}{0}) = 0$, but since $p \neq q$, $p_i = q_i = 0$ cannot be true for all $i$. Thus the sum maintains positive.
If $p_i = 0$ and $q_i > 0$, $p_i\log\qty(\dfrac{p_i}{0}) = 0$, but since $\sum{q_i} = 1 > 0$, $q_i = 0$ cannot be true for all $i$. Thus the sum maintains positive.

\vspace{3mm}
Compared to \textbf{Problem 4}, the difference is that $p \neq q$, and this results in $D_{KL}(p||q) = 0$ not being possible.
Since $\forall i \; p_i\log\qty(\dfrac{p_i}{q_i}) \geq 0$, for $D_{KL}(p||q)$ to be $0$, $p_i\log\qty(\dfrac{p_i}{q_i}) = 0$ should be satisfied at all $i$.
In that $\sum{p_i} = \sum{q_i} = 1$, this can only be satisfied when $p = q$.
Thus supposing $p \neq q$ eliminates the equality condition from \textbf{Problem 4}, resulting in a strict inequality to satisfy.

\section*{Problem 6}
\begin{enumerate}[leftmargin=*, label={(\alph*)}]
\item \begin{align*}
    \frac{\partial f_\theta(x)}{\partial u_i} &= \frac{\partial}{\partial u_i}\sum_{j=1}^{p}u_j\sigma\qty(a_j x + b_j) \\
    &= \sum_{j=1}^{p}\frac{\partial}{\partial u_i}u_j\sigma\qty(a_j x + b_j) = \frac{\partial}{\partial u_i}u_i\sigma\qty(a_i x + b_i) \\
    &= \sigma\qty(a_i x + b_i)
\end{align*}
Therefore
$$\nabla_{u}f_\theta(x) = \qty(\frac{\partial f_\theta(x)}{\partial u_1}, \cdots , \frac{\partial f_\theta(x)}{\partial u_p}) = \qty(\sigma\qty(a_1 x + b_1), \cdots, \sigma\qty(a_p x + b_p)) = \sigma(ax+b).$$
\item \begin{align*}
    \frac{\partial f_\theta(x)}{\partial b_i} &= \frac{\partial}{\partial b_i}\sum_{j=1}^{p}u_j\sigma\qty(a_j x + b_j) \\
    &= \sum_{j=1}^{p}\frac{\partial}{\partial b_i}u_j\sigma\qty(a_j x + b_j) = \frac{\partial}{\partial b_i}u_i\sigma\qty(a_i x + b_i) \\
    &= u_i\sigma'\qty(a_i x + b_i)
\end{align*}
Therefore
$$\nabla_{b}f_\theta(x) = \qty(\frac{\partial f_\theta(x)}{\partial b_1}, \cdots , \frac{\partial f_\theta(x)}{\partial b_p}) = \qty(u_1\sigma'\qty(a_1 x + b_1), \cdots, u_p\sigma'\qty(a_p x + b_p)) = \mathrm{diag}\qty(\sigma'(ax+b))u.$$
\item \begin{align*}
    \frac{\partial f_\theta(x)}{\partial a_i} &= \frac{\partial}{\partial a_i}\sum_{j=1}^{p}u_j\sigma\qty(a_j x + b_j) \\
    &= \sum_{j=1}^{p}\frac{\partial}{\partial a_i}u_j\sigma\qty(a_j x + b_j) = \frac{\partial}{\partial a_i}u_i\sigma\qty(a_i x + b_i) \\
    &= u_ix\sigma'\qty(a_i x + b_i)
\end{align*}
Therefore
$$\nabla_{a}f_\theta(x) = \qty(\frac{\partial f_\theta(x)}{\partial a_1}, \cdots , \frac{\partial f_\theta(x)}{\partial a_p}) = \qty(u_1x\sigma'\qty(a_1 x + b_1), \cdots, u_px\sigma'\qty(a_p x + b_p)) = \mathrm{diag}\qty(\sigma'(ax+b))ux.$$
\end{enumerate}
\end{document}