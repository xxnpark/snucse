\documentclass[10pt]{article}
%\usepackage[left=2.3cm,right=2.3cm,top=2.5cm,bottom=3cm,a4paper]{geometry}
\usepackage{fullpage}
\usepackage{setspace}
\setstretch{1.3}
\usepackage{amsmath,amssymb,amsthm,physics,units,mathtools,bm}
\usepackage{mdwlist}
\usepackage{paralist}
\setlength\parindent{0pt}
\usepackage{float}
\usepackage{xcolor}
\usepackage{algorithm,algpseudocode}
\usepackage{listings}
\usepackage[colorlinks=true]{hyperref}

\NewDocumentCommand{\code}{v}{%
\texttt{#1}%
}

\DeclareMathOperator*{\argmax}{argmax\,}
\DeclareMathOperator*{\argmin}{argmin\,}
\DeclareMathOperator*{\minimize}{minimize\,}
\DeclareMathOperator*{\maximize}{maximize\,}

\definecolor{dkgreen}{rgb}{0,0.6,0}
\definecolor{gray}{rgb}{0.5,0.5,0.5}
\definecolor{lgray}{RGB}{240,239,239}
\definecolor{mauve}{rgb}{0.58,0,0.82}

\lstset{
    % frame=tb,
    aboveskip=3mm,
    belowskip=3mm,
    showstringspaces=false,
    columns=flexible,
    basicstyle=\ttfamily,
    numbers=none,
    backgroundcolor=\color{lgray},
    numberstyle=\tiny\color{gray},
    keywordstyle=\color{blue},
    commentstyle=\color{dkgreen},
    stringstyle=\color{mauve},                
    keepspaces=true,
    breaklines=true,
    breakatwhitespace=true,
    tabsize=3
}

\begin{document}
\begin{center}
    {\LARGE MathDNN Homework 11} \\
\end{center}
\begin{flushright}
    Department of Computer Science and Engineering \\
    2021-16988 Jaewan Park
\end{flushright}

\section*{Problem 1}
\begin{enumerate}[(a)]
    \item Since $\log$ is a concave function, using Jensen's inequality, we obtain the following.
    \begin{align*}
        \mathrm{VLB}_{\theta, \phi}^{(K)}(x) &= \mathbb{E}_{Z_1, \cdots, Z_K \sim q_{\phi}(z|x)}\qty[\log\frac{1}{K}\sum_{k=1}^{K}\frac{p_{\theta}\qty(x \,|\, Z_k)p_Z\qty(Z_k)}{q_{\phi}\qty(Z_k \,|\, x)}] \\
        &\leq \log\qty(\mathbb{E}_{Z_1, \cdots, Z_K \sim q_{\phi}(z|x)}\qty[\frac{1}{K}\sum_{k=1}^{K}\frac{p_{\theta}\qty(x \,|\, Z_k)p_Z\qty(Z_k)}{q_{\phi}\qty(Z_k \,|\, x)}]) \\
        &= \log\qty(\frac{1}{K}\sum_{k=1}^{K}\mathbb{E}_{Z_k \sim q_{\phi}(z|x)}\qty[\frac{p_{\theta}\qty(x \,|\, Z_k)p_Z\qty(Z_k)}{q_{\phi}\qty(Z_k \,|\, x)}]) \\
        &= \log\qty(\frac{1}{K}\sum_{k=1}^{K}p_\theta(x)) = \log p_\theta(x)
    \end{align*}
    \item Using the given hint together with Jensen's inequality, we obtain the following.
    \begin{align*}
        \mathrm{VLB}_{\theta, \phi}^{(K)}(x) &= \mathbb{E}_{Z_{1}, \cdots, Z_{K} \sim q_{\phi}(z|x)}\qty[\log\frac{1}{K}\sum_{k=1}^{K}\frac{p_{\theta}\qty(x \,|\, Z_{k})p_Z\qty(Z_{k})}{q_{\phi}\qty(Z_{k} \,|\, x)}] \\
        &= \mathbb{E}_{Z_{i_1}, \cdots, Z_{i_M} \sim q_\phi(z|x)}\qty[\log\qty(\mathbb{E}_{I = \qty{i_1, \cdots, i_M}}\qty[\frac{1}{M}\sum_{m=1}^{M}\frac{p_\theta\qty(x|Z_{i_m})p_Z\qty(Z_{i_m})}{q_\phi\qty(Z_{i_m}|x)}])] \\
        &\geq \mathbb{E}_{Z_{i_1}, \cdots, Z_{i_M} \sim q_\phi(z|x)}\qty[\mathbb{E}_{I = \qty{i_1, \cdots, i_M}}\qty[\log\frac{1}{M}\sum_{m=1}^{M}\frac{p_\theta\qty(x|Z_{i_m})p_Z\qty(Z_{i_m})}{q_\phi\qty(Z_{i_m}|x)}]] \\
        &= \mathbb{E}_{I = \qty{i_1, \cdots, i_M}}\qty[\mathbb{E}_{Z_{i_1}, \cdots, Z_{i_M} \sim q_\phi(z|x)}\qty[\log\frac{1}{M}\sum_{m=1}^{M}\frac{p_\theta\qty(x|Z_{i_m})p_Z\qty(Z_{i_m})}{q_\phi\qty(Z_{i_m}|x)}]] \\
        &= \mathbb{E}_{I = \qty{i_1, \cdots, i_M}}\qty[\mathrm{VLB}_{\theta, \phi}^{(M)}(x)] = \mathrm{VLB}_{\theta, \phi}^{(M)}(x)
    \end{align*}
    \item We should choose $q_\phi$ powerful enough so that $q_\phi\qty(Z_k \,|\, x) = p_\theta\qty(Z_k \,|\, x)$ for all $k = 1, \cdots, K$.
    \begin{align*}
        \mathrm{VLB}_{\theta, \phi}^{(K)}(x) &= \mathbb{E}_{Z_1, \cdots, Z_K \sim q_\phi(z|x)}\qty[\log\frac{1}{K}\sum_{k=1}^{K}\frac{p_\theta\qty(x \,|\, Z_k)p_Z\qty(Z_k)}{q_\phi\qty(Z_k \,|\, x)}] \\
        &=  \mathbb{E}_{Z_1, \cdots, Z_K \sim q_\phi(z|x)}\qty[\log\frac{1}{K}\sum_{k=1}^{K}\frac{p_\theta\qty(x \,|\, Z_k)p_Z\qty(Z_k)}{p_\theta\qty(Z_k \,|\, x)}] \\
        &=  \mathbb{E}_{Z_1, \cdots, Z_K \sim q_\phi(z|x)}\qty[\log\frac{1}{K}\sum_{k=1}^{K}p_\theta(x)] = p_\theta(x)
    \end{align*}
\end{enumerate}

\section*{Problem 2}
\begin{enumerate}[(a)]
    \item Since $\log$ is a concave function, using Jensen's inequality, we obtain the following.
    \begin{align*}
        \log p_{\theta}\qty(X_i) &= \log\qty(\mathbb{E}_{Z \sim q_{\phi}(z|X_i)}\qty[\frac{p_{\theta}\qty(X_i \,|\, Z)r_\lambda\qty(Z)}{q_{\phi}\qty(Z \,|\, X_i)}]) \\
        &\geq \mathbb{E}_{Z \sim q_{\phi}(z|X_i)}\qty[\log\qty(\frac{p_{\theta}\qty(X_i \,|\, Z)r_\lambda\qty(Z)}{q_{\phi}\qty(Z \,|\, X_i)})] = \mathrm{VLB}_{\theta, \phi, \lambda}\qty(X_i)
    \end{align*}
    \item Gradients regarding $\theta$ and $\lambda$ can be easily derived as the follwing.
    $$\nabla_{\theta}\mathrm{VLB}_{\theta, \phi, \lambda}\qty(X_i) = \nabla_{\theta}\mathbb{E}_{Z \sim q_{\phi}(z|X_i)}\qty[\log\qty(\frac{p_{\theta}\qty(X_i \,|\, Z)r_\lambda\qty(Z)}{q_{\phi}\qty(Z \,|\, X_i)})] = \mathbb{E}_{Z \sim q_{\phi}(z|X_i)}\qty[\nabla_{\theta}\log{p_{\theta}\qty(X_i \,|\, Z)}]$$
    $$\nabla_{\lambda}\mathrm{VLB}_{\theta, \phi, \lambda}\qty(X_i) = \nabla_{\lambda}\mathbb{E}_{Z \sim q_{\phi}(z|X_i)}\qty[\log\qty(\frac{p_{\theta}\qty(X_i \,|\, Z)r_\lambda\qty(Z)}{q_{\phi}\qty(Z \,|\, X_i)})] = \mathbb{E}_{Z \sim q_{\phi}(z|X_i)}\qty[\nabla_{\lambda}\log{r_{\lambda}\qty(Z)}]$$
    For gradients on $\phi$, we can use the log-derivative trick.
    \begin{align*}
        \nabla_{\phi}\mathrm{VLB}_{\theta, \phi, \lambda}\qty(X_i) &= \nabla_{\phi}\mathbb{E}_{Z \sim q_{\phi}(z|X_i)}\qty[\log\qty(\frac{p_{\theta}\qty(X_i \,|\, Z)r_\lambda\qty(Z)}{q_{\phi}\qty(Z \,|\, X_i)})] = \nabla_{\phi}\int\log\qty(\frac{p_{\theta}\qty(X_i \,|\, z)r_\lambda\qty(z)}{q_{\phi}\qty(z \,|\, X_i)})q_{\phi}\qty(z \,|\, X_i)dz \\
        &= \int \qty(-\frac{\nabla_{\phi}q_\phi\qty(z \,|\, X_i)}{q_\phi\qty(z \,|\, X_i)}q_\phi\qty(z \,|\, X_i) + \log\qty(\frac{p_{\theta}\qty(X_i \,|\, z)r_\lambda\qty(z)}{q_{\phi}\qty(z \,|\, X_i)})\nabla_{\phi}q_\phi\qty(z \,|\, X_i))dz \\
        &= \int \log\qty(\frac{p_{\theta}\qty(X_i \,|\, z)r_\lambda\qty(z)}{q_{\phi}\qty(z \,|\, X_i)})\frac{\nabla_{\phi}q_\phi\qty(z \,|\, X_i)}{q_\phi\qty(z \,|\, X_i)}q_\phi\qty(z \,|\, X_i)dz \\
        &= \mathbb{E}_{Z \sim q_{\phi}(z|X_i)}\qty[\log\qty(\frac{p_{\theta}\qty(X_i \,|\, Z)r_\lambda\qty(Z)}{q_{\phi}\qty(Z \,|\, X_i)})\nabla_{\phi}\log{q_\phi\qty(Z \,|\, X_i)}]
    \end{align*}
    \item We can rewrite $\mathrm{VLB}$ as the following.
    \begin{align*}
        \mathrm{VLB}_{\theta, \phi, \lambda}\qty(X_i) &= \mathbb{E}_{Z \sim q_\phi\qty(z|X_i)}\qty[\log\qty(\frac{p_{\theta}\qty(X_i \,|\, Z)r_\lambda\qty(Z)}{q_{\phi}\qty(Z \,|\, X_i)})] \\
        &= \mathbb{E}_{Z \sim q_\phi\qty(z|X_i)}\qty[\log p_\theta\qty(X_i \,|\, Z)] - D_{\mathrm{KL}}\qty(q_\phi\qty(z \,|\, X_i) \:||\: r_\lambda(z))
    \end{align*}
    Then the first term can be calculated as
    \begin{align*}
        \mathbb{E}_{Z \sim q_\phi\qty(z|X_i)}\qty[\log p_\theta\qty(X_i \,|\, Z)] &= \mathbb{E}_{Z \sim \mathcal{N}\qty(\mu_\phi\qty(X_i), \Sigma_\phi\qty(X_i))}\qty[\log \mathcal{N}\qty(f_\theta(Z), \sigma^2I)] \\
        &= \mathbb{E}_{Z \sim \mathcal{N}\qty(\mu_\phi\qty(X_i), \Sigma_\phi\qty(X_i))}\Big[-\frac{1}{2}\qty(X_i - f_\theta(Z))^\intercal\qty(\sigma^2I)^{-1}\qty(X_i - f_\theta(Z)) \\
        &\hspace{8cm} - \frac{1}{2}\log\qty(\qty(2\pi)^k\qty|\sigma^2I|)\Big] \\
        &= -\frac{1}{2\sigma^2}\mathbb{E}_{Z \sim \mathcal{N}\qty(\mu_\phi\qty(X_i), \Sigma_\phi\qty(X_i))}\qty[\left\lVert X_i - f_\theta\qty(Z) \right\rVert^2] - \frac{k}{2}\log\qty(2\pi\sigma^2) \\
        &= -\frac{1}{2\sigma^2}\mathbb{E}_{\varepsilon \sim \mathcal{N}\qty(0, I)}\qty[\left\lVert X_i - f_\theta\qty(\mu_\phi\qty(X_i) + \sqrt{\Sigma_\phi\qty(X_i)}\varepsilon) \right\rVert^2] - \frac{k}{2}\log\qty(2\pi\sigma^2).
    \end{align*}
    Using the reparametrization trick simplifies the expectation term, and makes able the gradient of this first term be directly calculated.
    The second term also can be calculated as the following.
    \begin{align*}
        &D_{\mathrm{KL}}\qty(q_\phi\qty(z \,|\, X_i) \:||\: r_\lambda(z)) \\
        &\hspace{3mm} = \frac{1}{2}\qty(\mathrm{tr}\qty(\mathrm{diag}\qty(\lambda_2)^{-1}\Sigma_\phi\qty(X_i)) + \qty(\lambda_1 - \mu_\phi\qty(X_i))^\intercal\mathrm{diag}\qty(\lambda_2)^{-1}\qty(\lambda_1 - \mu_\phi\qty(X_i)) - k + \log\qty(\frac{\det\qty(\mathrm{diag}\qty(\lambda_2))}{\det\qty(\Sigma_\phi\qty(X_i))}))
    \end{align*}
    The gradient of the second term can also be directly calculated, so we can obtain the gradients via backpropagation.
\end{enumerate}

\section*{Problem 4}
\begin{enumerate}[(a)]
    \item Let $p_A = (p_{A1}, p_{A2}, p_{A3})$ and $p_B = (p_{B1}, p_{B2}, p_{B3})$.
    Then 
    $$\mathbb{E}_{p_A, p_B}\qty[\text{points for }B] = p_{A1}p_{B2} + p_{A2}p_{B3} + p_{A3}p_{B1} - p_{A1}p_{B3} - p_{A2}p_{B1} - p_{A3}p_{B2}.$$
    Suppose $p_A^* = (p_{A1}^*, p_{A2}^*, p_{A3}^*)$, $p_B^* = (p_{B1}^*, p_{B2}^*, p_{B3}^*)$ is the solution for the given problem. Then we have
    \begin{align*}
        &p_{A1}^*p_{B2} + p_{A2}^*p_{B3} + p_{A3}^*p_{B1} - p_{A1}^*p_{B3} - p_{A2}^*p_{B1} - p_{A3}^*p_{B2} \\
        &\hspace{2cm} \leq p_{A1}^*p_{B2}^* + p_{A2}^*p_{B3}^* + p_{A3}^*p_{B1}^* - p_{A1}^*p_{B3}^* - p_{A2}^*p_{B1}^* - p_{A3}^*p_{B2}^* \\
        &\hspace{4cm} \leq p_{A1}p_{B2}^* + p_{A2}p_{B3}^* + p_{A3}p_{B1}^* - p_{A1}p_{B3}^* - p_{A2}p_{B1}^* - p_{A3}p_{B2}^*.
    \end{align*}
    for all $p_A, p_B \in \Delta^3$.
    If $p_A^* = p_B^* = \qty(\dfrac{1}{3}, \dfrac{1}{3}, \dfrac{1}{3})$, all three terms are 0, so it is a solution of the problem.
    Now we should show that this is the only solution for the problem. 
    Suppose $p_A^* \neq \qty(\dfrac{1}{3}, \dfrac{1}{3}, \dfrac{1}{3})$ and generally let $p_{A1}^* < p_{A2}^*$. 
    Now substitute $p_A = \qty(\dfrac{1}{3}, \dfrac{1}{3}, \dfrac{1}{3})$ and $p_B = (0, 0, 1)$, then
    \begin{align*}
        p_{A1}^*p_{B2} + p_{A2}^*p_{B3} + p_{A3}^*p_{B1} - p_{A1}^*p_{B3} - p_{A2}^*p_{B1} - p_{A3}^*p_{B2} &= p_{A2}^* - p_{A1}^* > 0 \\
        p_{A1}p_{B2}^* + p_{A2}p_{B3}^* + p_{A3}p_{B1}^* - p_{A1}p_{B3}^* - p_{A2}p_{B1}^* - p_{A3}p_{B2}^* &= 0
    \end{align*}
    so the inequality becomes false. 
    Similarly, suppose $p_B^* \neq \qty(\dfrac{1}{3}, \dfrac{1}{3}, \dfrac{1}{3})$ and generally let $p_{B1}^* < p_{B2}^*$. 
    Now substitute $p_A = (0, 0, 1)$ and $p_B = \qty(\dfrac{1}{3}, \dfrac{1}{3}, \dfrac{1}{3})$, then
    \begin{align*}
        p_{A1}^*p_{B2} + p_{A2}^*p_{B3} + p_{A3}^*p_{B1} - p_{A1}^*p_{B3} - p_{A2}^*p_{B1} - p_{A3}^*p_{B2} &= 0 \\
        p_{A1}p_{B2}^* + p_{A2}p_{B3}^* + p_{A3}p_{B1}^* - p_{A1}p_{B3}^* - p_{A2}p_{B1}^* - p_{A3}p_{B2}^* &= p_{B1}^* - p_{B2}^* < 0
    \end{align*}
    so the inequality also becomes false. 
    Therefore always $p_A^* = p_B^* = \qty(\dfrac{1}{3}, \dfrac{1}{3}, \dfrac{1}{3})$, so it is the unique solution for the problem.
    \item If $B$ chooses $p_B$ as given, the expected points for $B$ is always 0 regardless of $A$, so $A$ can choose any strategy.
    However, if $B$ chooses strategies other than $p_B = \qty(\dfrac{1}{3}, \dfrac{1}{3}, \dfrac{1}{3})$, choosing any strategy may not be optimal for $A$.
    Choosing $p_B = (1, 0, 0), (0, 1, 0), (0, 0, 1)$ results in $\mathbb{E}_{p_A, p_B}\qty[\text{points for }B] > 0$ each when $A$ chooses strategies such that $p_{A3} > p_{A2}$, $p_{A1} > p_{A3}$, $p_{A2} > p_{A1}$.
\end{enumerate}

\end{document}