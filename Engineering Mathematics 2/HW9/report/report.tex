\documentclass{article}
\usepackage{setspace}
\usepackage{bm}
\usepackage{amsmath, amsfonts, amssymb, physics}
\usepackage{graphicx}
\usepackage{mdwlist}
\usepackage[colorlinks=true]{hyperref}
\usepackage{geometry}
\geometry{margin=1in}
\usepackage{palatino}
\usepackage{hyperref}
\usepackage{algorithm,algpseudocode}
\usepackage{listings}
\usepackage{paralist}
\usepackage{todonotes}
\setlength{\marginparwidth}{2.15cm}
\usepackage{tikz}
\usetikzlibrary{positioning,shapes,backgrounds}
\setlength\parindent{0pt}

\definecolor{dkgreen}{rgb}{0,0.6,0}
\definecolor{gray}{rgb}{0.5,0.5,0.5}
\definecolor{lgray}{RGB}{240,239,239}
\definecolor{mauve}{rgb}{0.58,0,0.82}

\lstset{
    % frame=tb,
    aboveskip=3mm,
    belowskip=3mm,
    showstringspaces=false,
    columns=flexible,
    basicstyle=\ttfamily,
    numbers=none,
    backgroundcolor=\color{lgray},
    numberstyle=\tiny\color{gray},
    keywordstyle=\color{blue},
    commentstyle=\color{dkgreen},
    stringstyle=\color{dkgreen},                
    keepspaces=true,
    breaklines=true,
    breakatwhitespace=true,
    tabsize=3
}

\begin{document}
\vspace*{-1.5cm}
{\centering \vbox{%
\vspace{2mm}
\large
Engineering Mathematics 2 \hfill
\\
Seoul National University
\\[4mm]
Homework 9 \\
\textbf{2021-16988 Jaewan Park} \\[0.8mm]
}}
\par\noindent\rule{\textwidth}{0.5pt}

\setstretch{1.3}

\section*{Exercise 9.2} 
We have $X \sim \mathcal{N}(0, 1)$, and its density function is $f_X(x) = \dfrac{1}{\sqrt{2\pi}}\exp\qty(-x^2/2)$.
Therefore when $n \geq 2$,
\begin{align*}
    \mathbf{E}\qty[X^n] &= \int_{-\infty}^{\infty}\frac{1}{\sqrt{2\pi}}e^{-x^2/2}x^ndx \\
    &= \qty[-\frac{1}{\sqrt{2\pi}}e^{-x^2/2}x^{n-1}]_{-\infty}^{\infty} + \int_{-\infty}^{\infty}\frac{1}{\sqrt{2\pi}}(n-1)e^{-x^2/2}x^{n-2}dx \\
    &= (n-1)\mathbf{E}\qty[X^{n-2}].
\end{align*}
For even $n \geq 2$, 
\begin{align*}
    \mathbf{E}\qty[X^n] = (n-1)\mathbf{E}\qty[X^{n-2}] = \cdots = (n-1)(n-3)\cdots{1}\mathbf{E}\qty[X^0] = (n-1)(n-3)\cdots{1} \geq 1
\end{align*}
For odd $n \geq 3$, 
\begin{align*}
    \mathbf{E}\qty[X^n] = -(n-1)\mathbf{E}\qty[X^{n-2}] = \cdots = (-1)^{(n-1)/2}(n-1)(n-3)\cdots{2}\mathbf{E}\qty[X^1] = 0
\end{align*}
Since $\mathbf{E}\qty[X] = 0$, we can say $\mathbf{E}\qty[X^n] = 0$ for all odd $n \geq 1$.

\section*{Exercise 9.3}
We can calculate the covariance as the following.
\begin{align*}
    \mathbf{Cov}\qty(Y_i, Y_j) &= \mathbf{E}\qty[\qty(Y_i - \mathbf{E}\qty[Y_i])\qty(Y_j - \mathbf{E}\qty[Y_j])] = \mathbf{E}\qty[\qty(a_{i1}X_1 + \cdots + a_{in}X_n)\qty(a_{j1}X_1 + \cdots + a_{jn}X_n)] \\
    &= \mathbf{E}\qty[\sum_{1 \leq p, q \leq n}a_{ip}a_{jq}X_pX_q] = \sum_{1 \leq p, q \leq n}a_{ip}a_{jq}\mathbf{E}\qty[X_pX_q] \\
    &= \sum_{k=1}^{n}a_{ik}a_{jk}\mathbf{E}\qty[X_k^2] = \sum_{k=1}^{n}a_{ik}a_{jk}
\end{align*}

\section*{Exercise 9.4}
\begin{enumerate}[(a)]
    \item For $n$ datapoints of $X$ and $Y$, let $\mathbf{u} = \mathbf{X} - \mathbf{E}\qty[X]$, $\mathbf{v} = \mathbf{Y} - \mathbf{E}\qty[Y]$ where $\mathbf{X}$ and $\mathbf{Y}$ are the vectors of the datapoints.
    Then we obtain the following.
    $$\mathbf{Cov}\qty(X, Y) = \mathbf{E}\qty[(X - \mathbf{E}\qty[X])(Y - \mathbf{E}\qty[Y])] = \frac{1}{n-1}\sum_{i=1}^{n}\qty(X_i - \mathbf{E}\qty[X])(Y_i - \mathbf{E}\qty[Y]) = \frac{1}{n-1}\mathbf{u}\cdot\mathbf{v}$$
    $$\mathbf{Var}\qty(X) = \mathbf{E}\qty[(X - \mathbf{E}\qty[X])^2] = \frac{1}{n-1}\sum_{i=1}^{n}\qty(X_i - \mathbf{E}\qty[X])^2 = \frac{1}{n-1}\lVert\mathbf{u}\rVert^2$$
    $$\mathbf{Var}\qty(Y) = \mathbf{E}\qty[(Y - \mathbf{E}\qty[Y])^2] = \frac{1}{n-1}\sum_{i=1}^{n}\qty(Y_i - \mathbf{E}\qty[Y])^2 = \frac{1}{n-1}\lVert\mathbf{v}\rVert^2$$
    From the Cauchy-Schwarz inequality we know that $\qty|\mathbf{u}\cdot\mathbf{v}| \leq \lVert\mathbf{u}\rVert\lVert\mathbf{v}\rVert$, so 
    $$\qty|\mathbf{Cov}\qty(X, Y)| \leq \sqrt{\mathbf{Var}\qty(X)}\sqrt{\mathbf{Var}\qty(Y)}.$$
    Therefore
    $$\qty|\rho_{XY}| = \frac{\qty|\mathbf{Cov}\qty(X, Y)|}{\sigma_X\sigma_Y} \leq 1.$$
    \item Since $\mathbf{Cov}\qty(X, Y) = \mathbf{E}\qty[XY] - \mathbf{E}\qty[X]\mathbf{E}\qty[Y]$, if $X$ and $Y$ are independent then $\mathbf{Cov}\qty(X, Y) = 0$ and $\rho_{XY} = 0$.
    \item Let $X$ be a random variable that is either $-1$ or $1$ with probability $0.5$, and $Y$ a random variable that is $0$ if $X = -1$ and either $-1$ or $1$ with probability $0.5$ if $X = 1$.
    Then both $X$ and $Y$ have $0$ mean, and 
    $$\mathbf{E}\qty[XY] = (-1) \times 0 \times 0.5 + 1 \times (-1) \times 0.25 + 1 \times 1 \times 0.25 = 0.$$
    Therefore $\mathbf{Cov}\qty(X, Y) = 0$ and $\rho_{XY} = 0$.
\end{enumerate}

\section*{Exercise 9.6}
The following Python code gives us the result, and the upper bound is approximately $0.02$.
\begin{lstlisting}[language=Python]
from statistics import NormalDist
import numpy as np
import matplotlib.pyplot as plt

N = 10000

plt.title("Exercise 9.6")
plt.xlabel("z")
plt.ylabel("|P(X <= Z) - P(Y <= Z)|")

for z in np.linspace(-3, 3, 1000):
    x_le_z_count = 0
    for _ in range(N):
        x_le_z_count += int((sum(np.random.uniform(0, 1, 12)) - 6) <= z)
    p_x_le_z = x_le_z_count / N
    p_y_le_z = NormalDist(0, 1).cdf(z)
    plt.scatter(z, abs(p_x_le_z - p_y_le_z), color='black')

plt.savefig("plot.png")
\end{lstlisting}
\begin{figure}[H]
    \centering
    \includegraphics[width=0.8\textwidth]{"plot.png"}
\end{figure}

\section*{Exercise 9.14}
\begin{enumerate}[(a)]
    \item We can compare the distribution functions as the following.
    \begin{align*}
        \mathrm{Pr}\qty(Y \leq y) &= \mathrm{Pr}\qty(XZ \leq y) = \frac{1}{2}\mathrm{Pr}\qty(X \leq y) + \frac{1}{2}\mathrm{Pr}\qty(X \geq -y) = \frac{1}{2}\mathrm{Pr}\qty(X \leq y) + \frac{1}{2}\mathrm{Pr}\qty(X \leq y) = \mathrm{Pr}\qty(X \leq y)
    \end{align*}
    Therefore $Y$ has the same distribution as $X$.
    \item Since $X$ and $Y$ both follow standard normal distributions, we know in the case of the following,  
    \begin{align*}
        \mathrm{Pr}\qty(X \leq -2, Y \leq -1) &= \mathrm{Pr}\qty(X \leq -2, XZ \leq -1) \\
        &= \frac{1}{2}\mathrm{Pr}\qty(X \leq -2, X \leq -1) + \frac{1}{2}\mathrm{Pr}\qty(X \leq -2, X \geq 1) \\
        &= \frac{1}{2}\mathrm{Pr}\qty(X \leq -2) \neq \mathrm{Pr}\qty(X \leq -2)\mathrm{Pr}\qty(Y \leq -1).
    \end{align*}
    the joint distribution functions do not match.
    Therefore $X$ and $Y$ are dependent.
    \item Let $B \sim \mathit{Ber}\qty(\dfrac{1}{2})$ random variable, then $Y = X(2B-1)$. 
    If $X$ and $Y$ are jointly normal, $X+Y$, which is a linear combination of the two, should also be normally distributed. 
    Since $X+Y = 2BX$, 
    $$\mathrm{Pr}\qty(X+Y \leq k) = \mathrm{Pr}\qty(2BX \leq k) = \frac{1}{2}\mathrm{Pr}\qty(2X \leq k) + \frac{1}{2}\mathrm{Pr}\qty(0 \leq k)$$
    so $X+Y$ is a combination of $2X \sim \mathcal{N}(0, 4)$ and a fixed point $0$. Therefore $X+Y$ is not normally distributed.
    \item 
    The distribution and density functions of $XY$ are
    \begin{align*}
        F_{XY}(k) &= \mathrm{Pr}\qty(XY \leq k) = \mathrm{Pr}\qty(X^2Z \leq k) = \frac{1}{2}\mathrm{Pr}\qty(X^2 \leq k) + \frac{1}{2}\mathrm{Pr}\qty(X^2 \geq -k) \\
        &= \begin{cases}
            F_{X}\qty(\sqrt{k}) & (k \geq 0) \\
            1 - F_{X}\qty(\sqrt{-k}) & (k < 0)
        \end{cases} \\
        f_{XY}(k) &= \frac{d}{dk}F_{XY}(k) = \frac{1}{2\sqrt{|k|}}f_{X}\qty(\sqrt{|k|}).
    \end{align*}
    Therefore the correlation coeffecient is
    \begin{align*}
        \rho_{XY} &= \frac{\mathbf{Cov}\qty(X, Y)}{\sigma_X\sigma_Y} = \mathbf{E}\qty[XY] \\
        &= \int_{-\infty}^{\infty}k\cdot\frac{1}{2\sqrt{|k|}}f_{X}\qty(\sqrt{|k|})dk = \int_{-\infty}^{0}-\frac{1}{2}\sqrt{-k}f_{X}\qty(\sqrt{-k})dk + \int_{0}^{\infty}\frac{1}{2}\sqrt{k}f_{X}\qty(\sqrt{k})dk \\
        &= 0.
    \end{align*}
\end{enumerate}

\end{document}