\documentclass[10pt]{article}
\usepackage[left=2.3cm,right=2.3cm,top=2.5cm,bottom=3cm,a4paper]{geometry}
%\usepackage{fullpage}
\usepackage{setspace}
\setstretch{1.2}
\usepackage{amsmath,amssymb,amsthm,physics,units}
\usepackage[shortlabels]{enumitem}
\setlength\parindent{0pt}
\usepackage{float}
\usepackage{algorithm,algpseudocode}
\usepackage[shortlabels]{enumitem}
\usepackage{hyperref}
\hypersetup{
    colorlinks=true,
    linkcolor=black,
    filecolor=black,      
    urlcolor=black,
    pdftitle={Overleaf Example},
    pdfpagemode=FullScreen,
}

\begin{document}
\vspace{4mm}
\begin{center}
    {\LARGE Discrete Mathematics Problem Set 3} \\
\end{center}
\begin{flushright}
    Department of Computer Science and Engineering \\
    2021-16988 Jaewan Park
\end{flushright}

\section*{Problem 1}

Suppose there exists at least one positive integer $n$ such that $P(n)$ is false. 
If $S$ is the set of all such `$n$'s, $S$ is a nonempty set of positive integers.
Thus according to the well-ordering property, $S$ has a least element.
Let this element $m$.

\vspace{3mm}
Considering the basis step, $m > b$ since $P(m)$ is false.
Also he contrapositive of the inductive step states the following.
\begin{center}
   If $P(k+b)$ is false, $P(k) \wedge P(k+1) \wedge \cdots \wedge P(k+b-1)$ is false.
\end{center}
Therefore, since $P(m)$ is false, $P(m-b) \wedge P(m-b+1) \wedge \cdots \wedge P(m-1)$ is false.
This gives that at least one of $P(m-b)$, $P(m-b+1)$, $\cdots$, $P(m-1)$ is false. (These are all valid positive integers since $m-b > 0$.)
However, since $m$ is the least element of $S$, there shouldn't be an integer smaller than $m$ that makes $P(n)$ false.
Therefore this is a contradiction, and the given form of mathematical induction is valid to prove that $P(n)$ is true for all positive integers $n$.

\section*{Problem 2}

First, let's consider cases when $n$ is a power of 2. Let $n = 2^m$.

\vspace{3mm}
Sort the numbers first in order of mod $2^m$, then mod $2^{m-1}$, and so on until mod $2^1$. (When the modulo values are equal, leave them in the order of the previous sort.)
For example, for $n=16$, we can sort as :
\begin{center}
    1 2 3 4 5 6 7 8 9 10 11 12 13 14 15 16 \\
    $\downarrow$ \\
    1 9 2 10 3 11 4 12 5 13 6 14 7 15 8 16 \\
    $\downarrow$ \\
    1 9 5 13 2 10 6 14 3 11 7 15 4 12 8 16 \\
    $\downarrow$ \\
    1 9 5 13 3 11 7 15 2 10 6 14 4 12 8 16
\end{center}
Now if we divide the numbers into $2^k \; (l = 1, \cdots , m)$ groups with equal size starting from the front, all groups have numbers with same mod $2^k$ values, and the groups each other have different mod $2^k$ values.
Let's call each group `Group$(k, p)$' where all elements' mod $2^k$ values are $p$.

\vspace{3mm}
For $\forall k$ and $0 \leq \forall p, q < 2^k \; (p \neq q)$, arbitrary integers $a$ and $b$ where $a \equiv p \; ({\rm mod} \; 2^k)$, $b \equiv q \; ({\rm mod} \; 2^k)$ shows
$$\frac{a+b}{2} \equiv \frac{p+q}{2} \not\equiv p {\; \rm or \;} q \; ({\rm mod} \; 2^k)$$
Therefore, when two groups Group$(k, p)$ and Group$(k, q)$ are adjacent, choosing two numbers from each group cannot have their average between them. 
Only choosing two numbers from the same group can have a possibility of having their average between them.

\vspace{3mm}
Considering the sorted list of numbers, suppose there is a pair of numbers where their average exists between them. 
However first, numbers of Group(1, 1) and Group(1, 0) don't have such a pair in between them, so we should find the pair inside each group.
But Group(1,1) is consisted of Group(2,1) and Group(2,3) so we should also find the pair inside each group.
This repeats until Group(k, 1), $\cdots$, Group$(k, 2^k-1)$, Group$(k, 0)$. So we cannot find such a pair in the sorted list of numbers.
Therefore if $n$ is a power of 2, it is possible to sort the numbers in a row so that the average of any two number never appears between them.

\vspace{3mm}
Now if $n$ is not a power of 2, consider the smallest power of 2 larger than $n$.
We can sort the numbers so that it satisfies the condition, and remove all numbers larger than $n$.
Before removing, all pairs of numbers in the sorted list didn't have their average between them.
Therefore after removing some numbers all pairs of numbers will also not have their average between them.
Therefore we can find a an arrangement of numberes for any positive integer $n$ so that the average of any two number never appears between them.

\section*{Problem 3}

We must make $n-1$ breaks to break the bar into $n$ separate squares. We can prove this using strong induciton.

\vspace{3mm}
First, consider the case when $n=1$. Then the entire bar is already a separate square, so it requires $0 \; (=n-1)$ breaks.

\vspace{3mm}
Let's assume the proposition is true for $n=1, \; n=2, \; \cdots , \; n=k$.
When $n=k+1$, we can make a first valid break, and say it results in two rectangular bars of size $m$ and $k+1-m$.
The assumption gives that each bars need $m-1$, $k-m$ breaks to make it into separate pieces.
Therefore a bar of $k+1$ squares needs the following number of breaks in total.
\begin{align*}
    {\rm Number \; of \; Breaks} &= 1 + (m-1) + (k-m) \\
    &= k = (k+1) - 1
\end{align*}
Therefore the proposition is true for $n=k+1$, and using strong induction, the proposition is true for any positive integer $n$.

\section*{Problem 4}

First, let's count the number of strings that contain five consecutive 0s.
Consider the first index(indexes given 1 - 10) of the 0 with four consecutive 0s continued.
It can have values among 1, 2, 3, 4, 5, 6.

\vspace{3mm}
If the index is 1, indexes 1 - 5 are assigned 0, and indexes 6 - 10 can have any value among 0 or 1.
The total number of possibilities is $2^5 = 32$.

\vspace{3mm}
If the index is 2 or 3 or 4 or 5 or 6, five indexes starting from that are assigned 0 and the index before that is assigned 1. The remaining four indexes can have any value among 0 or 1.
The total number of possibilities is $2^4 = 16$ for each.

\vspace{3mm}
Therefore the number of strings that contain five consecutive 0s is $32 + 16 \times 5 = 112$.
Similarly, the number of strings that contain five consecutive 1s is also 112.
However the strings `0000011111' and `1111100000' are counted in both cases, so we should exclude the repeated counts.
Therefore the total number of strings is $112 + 112 - 2 = 222$.

\section*{Problem 5}

The number of multiples of $p$ not exceeding $n$ is $\dfrac{n}{p} = qr$.
Similarly the number of multiples of $q$ and $r$ not exceeding $n$ are $pr$, $pq$ each.

\vspace{3mm}
Multiples of $pq$, $qr$, $rp$ are counted twice each, so we should exclude the repeated counts.
The number of multiples of $pq$, $qr$, $rp$ not exceeding $n$ are $r, p, q$ each.

\vspace{3mm}
Now multiples of $pqr$ are excluded, so we should add the number of these.
The number of multiples of $pqr$ not exceeding $n$ is 1.

\vspace{3mm}
Therefore using the principle of inclusion-exclusion, the number of positive integers not exceeding $n$ that are relatively prime to $n$ is $pq + qr + rp - p - q - r + 1$.

\section*{Problem 6}
Consider a situation where there is a group of $n$ mathematics professors and another group of $n$ computer science professors.
We would like to select a committee of $n$ members from any group, instead the chairperson is a mathematics professor.
We can count this in two ways.

\vspace{3mm}
First, consider all cases where $k$ mathematics professors and $n-k$ computer science professors are selected ($k = 0, 1, \cdots, n$).
For each case, selecting mathematics professors gives ${n \choose k}$ possibilities and selecting computer science professors gives ${n \choose n-k}$ possibilities. Also selecting the chairperson gives $k$ possibilities, since there are $k$ mathematics professors.
Therefore the total number of selections is 
$$\sum_{k=0}^{n}k{n \choose k}{n \choose n - k} = \sum_{k=0}^{n}k{n \choose k}^2$$

\vspace{3mm}
Second, consider choosing one chairperson from the $n$ mathematics professors, and then choosing $n-1$ professors from the remaining $2n-1$ professors.
Choosing one chairperson has $n$ possibilities and choosing other professors gives $n{2n-1 \choose n-1}$ possibilities.
Therefore the total number of selections is 
$$n{2n-1 \choose n-1}$$

$$\therefore \sum_{k=0}^{n}k{n \choose k}^2 = n{2n-1 \choose n-1}$$

\section*{Problem 7}

Let's define a random variable $X$ as the number of balls tossed until every bin contains a ball.
Also define a random varaiable $X_k$ as the number of balls tossed until a ball goes into a unfilled bin, after $k$ bins contain a ball.
The definition of each varaibles gives :
$$X = \sum_{k=0}^{b-1}X_k$$

$X_k$ can be any positive integer. Since the probability of a single toss to fall in one of the $k$ bins is $\dfrac{k}{b}$, the probability of $X_k$ having a certain value $n$ is :
$$P(X_k = n) = \qty(\frac{k}{b})^{n-1}\qty(1-\frac{k}{b})$$
Therefore $X_k$ has a geometric distribution with parameter $1 - \dfrac{k}{b}$, so the expectation of $X_k$ is :
\begin{align*}
    E\qty(X_k) &= \sum_{n=1}^{\infty}n\qty(\frac{k}{b})^{n-1}\qty(1-\frac{k}{b}) \\
    &= \frac{b}{b-k}
\end{align*}

Therefore the expectation of $X$ is :
\begin{align*}
    E(X) &= E\qty(\sum_{k=0}^{b-1}X_k) \\
    &= \sum_{k=0}^{b-1}E\qty(X_k) = \sum_{k=0}^{b-1}\frac{b}{b-k} \\
    &= b\qty(1 + \frac{1}{2} + \cdots + \frac{1}{b})
\end{align*}

\section*{Problem 8}

\begin{enumerate}[leftmargin=*]
    \item Let the event where 0 is sent as S, and the event where 0 is received R
    The probability that 0 is received is given by the probability of 0 sent mulitplied by the probability of 0 received correctly when it is sent.
    \begin{align*}
        P(R) &= P(S)P(R|S) + P\qty(\overline{S})P\qty(R|\overline{S}) \\
        &= \frac{2}{3}\cdot 0.9 + \frac{1}{3}\cdot 0.2 \\
        &= \frac{2}{3}
    \end{align*}
    \item The probability is :
    \begin{align*}
        P(S|R) &= \frac{P(S)P(R|S)}{P(S)P(R|S) + P\qty(\overline{S})P\qty(R|\overline{S})} \\
        &= \frac{\dfrac{2}{3}\cdot 0.9}{\dfrac{2}{3}} \\
        &= 0.9
    \end{align*}
\end{enumerate}

\section*{Problem 9}

Let's define a random variable $X$ as the number of tails that come up when the coin is tossed $n$ times.
$X$ has a binomial distribution, and $P(X=k) = C(n, k)\cdot{0.8}^k{0.2}^{n-k}$.
Therefore the expectation and variance of $X$ are :
$$E(X) = \sum_{k=0}^{n}k{n \choose k}{0.8}^k{0.2}^{n-k} = 0.8n$$
$$V(X) = \sum_{k=0}^{n}\qty(k-E(X))^2{n \choose k}{0.8}^k{0.2}^{n-k} = 0.16n$$
Therefore Chebyshev's inequality gives :
$$P\qty(\qty|X(s) - 0.8n| \geq \sqrt{n}) \leq \frac{0.16n}{n} = 0.16$$

\section*{Problem 10}

If random variables $X$ and $Z$, $Y$ and $Z$ are independent, we can say $E(XZ) = E(X)E(Z)$, $E(YZ) = E(Y)E(Z)$.
This gives $E((X+Y)Z) = E(XZ + YZ) = E(XZ) + E(YZ) = E(X)E(Z) + E(Y)E(Z) = (E(X) + E(Y))E(Z) = E(X+Y)E(Z)$.
Therefore $X+Y$ and $Z$ are independent.

\vspace{3mm}
Likewise, all sums of $X_i (i = 1, 2, \cdots, 10)$ are independent with each other.

\vspace{3mm}
For all two independent random varaiables $X$ and $Y$, we can derive the following.
\begin{align*}
    V(X + Y) &= E\qty(\qty(X + Y)^2) - {E\qty(X + Y)}^2 \\
    &= E\qty(X^2 + 2XY + Y^2) - \qty({E\qty(X)}^2 + 2E\qty(X)E\qty(Y) + {E\qty(Y)}^2) \\
    &= E\qty(X^2) + 2E\qty(X)E\qty(Y) + E\qty(Y^2) - {E\qty(X)}^2 - 2E\qty(X)E\qty(Y) - {E\qty(Y)}^2 \\
    &= E\qty(X^2) - {E\qty(X)}^2 + E\qty(Y^2) - {E\qty(Y)}^2 \\
    &= V\qty(X) + V\qty(Y)
\end{align*}

Therefore we can show:
\begin{align*}
    V(X_1 + X_2 + \cdots X_n) &= V(X_1 + X_2 + \cdots X_{n-1}) + V(X_n) \\
    &= V(X_1 + X_2 + \cdots X_{n-2}) + V(X_{n-1}) + V(X_n) \\
    &= V(X_1 + X_2 + \cdots X_{n-3}) + V(X_{n-2}) + V(X_{n-1}) + V(X_n) \\
    &= \cdots \\
    &= V(X_1) + V(X_2) + \cdots + V(X_n)
\end{align*}

\section*{References}
\begin{enumerate}[leftmargin=*, label={[\arabic*]}]
    \item `Ordering of natural numbers', \textit{Mathematics Stack Exchange}, 2021. Available: \url{https://math.stackexchange.com/questions/1403694/ordering-of-natural-numbers}.
    \item `Coupon collector's problem', \textit{Wikipedia}, 2021. Available: \url{https://en.wikipedia.org/wiki/Coupon_collector%27s_problem}.
\end{enumerate}

\vspace{0.3cm}
\textbf{Usage of References}

\vspace{0.2cm}
Problem 2 : I referred to the solution of the same problem in [1]. \\
Problem 7 : I referred to the solution of a similar problem in [2].

\end{document}