\documentclass[10pt]{article}
\usepackage{fullpage}
\usepackage{changepage}
\usepackage{setspace}
\setstretch{1.2}
\usepackage{amsmath,amssymb,physics}
\usepackage{enumitem}
\setlength\parindent{0pt}
\usepackage{float}
\usepackage{xcolor}

\begin{document}
\vspace{4mm}
\begin{center}
  {\LARGE Discrete Mathematics Problem Set 1} \\
\end{center}
\begin{flushright}
  Department of Computer Science and Engineering \\
  2021-16988 Jaewan Park
\end{flushright}

\section*{Problem 1}
\begin{enumerate}
  \item $\neg q \wedge r \rightarrow p$
  \item $p \rightarrow \neg q \wedge r$
  \item If you can install the program your computer has at least 2GB of RAM and 20GB of free disk space.
  \item $\neg p \rightarrow q \vee \neg r$
  \item If you can't install the program your computer has less than 2GB of RAM or 20GB of free disk space.
\end{enumerate}

\section*{Problem 2}
We can construct a truth table.
\begin{table}[H]
  \centering
  \renewcommand{\arraystretch}{1.5}
  \begin{tabular}{|c|c|c|c|c|c|c|}
    \hline
    $P$ & $Q$ & $R$ & $P \rightarrow Q$ & $(P \rightarrow Q) \rightarrow R$ & $Q \rightarrow R$ & $P \rightarrow (Q \rightarrow R)$ \\ \hline
    True & True & True & True & True & True & True \\ \hline
    True & True & False & True & False & False & False \\ \hline
    True & False & True & False & True & True & True \\ \hline
    True & False & False & False & True & True & True \\ \hline
    False & True & True & True & True & True & True \\ \hline
    \bfseries False & \bfseries True & \bfseries False & \bfseries True & \bfseries \textcolor{red}{False} & \bfseries False & \bfseries \textcolor{red}{True} \\ \hline
    False & False & True & True & True & True & True \\ \hline
    \bfseries False & \bfseries False & \bfseries False & \bfseries True & \bfseries \textcolor{red}{False} & \bfseries True & \bfseries \textcolor{red}{True} \\
    \hline
  \end{tabular}
\end{table}

As it shows, when $(P, Q, R) = $ (False, True, False) or (False, False, False), $(P \rightarrow Q) \rightarrow R$ and $P \rightarrow (Q \rightarrow R)$ represent different values.
Therefore the two expressions are logically inequivalent.

\section*{Problem 3}
For every $i, j \; (i, j = 1, 2, \cdots , 9)$, at least one of the propositions ``The number at cell $(i, j)$ is $n$" $\left(= p(i, j, n)\right)$ when $n = 1, 2, \cdots , 9$ should be true. Therefore the compound proposition is:
$$\bigwedge_{i=1}^{9}\bigwedge_{j=1}^{9}\bigvee_{n=1}^{9}p(i, j, n)$$

\section*{Problem 4}
\paragraph{(a)}
The 5 atomic statements are:

\vspace{0.3cm}
\hspace{0.2cm} $L$ : This house is next to a lake. 

\hspace{0.2cm} $K$ : The treasure is in the kitchen. 

\hspace{0.2cm} $E$ : The tree in the front yard is an elm. 

\hspace{0.2cm} $P$ : The treasure is buried under the flagpole. 

\hspace{0.2cm} $G$ : The treasure is in the garage.

\vspace{0.3cm}
Then the above compound statements can be expressed as:
\begin{enumerate}
  \item $L \rightarrow \neg K$
  \item $E \rightarrow K$
  \item $L$
  \item $E \vee \neg P$
  \item $P \oplus G \Leftrightarrow P \wedge \neg G \vee \neg P \wedge G$
\end{enumerate}

\paragraph{(b)} The treasure is in the garage. The proposition $G$ is true.

The proof can be written in two ways, either using the $\oplus$ operator or not. 
Using it gives:
\begin{table}[H]
  \hspace{0.3cm}
  \renewcommand{\arraystretch}{1.3}
  \begin{tabular}{ccl}
    (1) & $L \rightarrow \neg K$ & Premise \\
    (2) & $L$ & Premise \\
    (3) & $\neg K$ & Modus Ponens from (1) and (2) \\
    (4) & $E \rightarrow K$ & Premise \\
    (5) & $\neg E$ & Modus Tollens from (3) and (4) \\
    (6) & $E \vee \neg P$ & Premise \\
    (7) & $\neg P$ & Disjunctive Syllogism from (5) and (6) \\
    (8) & $P \oplus G$ & Premise \\
    (9) & $G$ & Definition of $\oplus$(exclusive or) using (8)
  \end{tabular}
\end{table}
Thus the treasure is in the garage. If we do not use the operator, lines (8) and (9) should be changed as:
\begin{table}[H]
  \hspace{0.3cm}
  \renewcommand{\arraystretch}{1.3}
  \begin{tabular}{ccl}
    (8) & $P \wedge \neg G \vee \neg P \wedge G$ & Premise \\
    (9) & $\neg P \vee G$ & Addition from (7) \\
    (10) & $\neg(P \wedge \neg G)$ & De Morgan's Law from (9) \\
    (11) & $\neg P \wedge G$ & Disjunctive Syllogism from (8) and (9) \\
    (12) & $G$ & Simplification from (11)
  \end{tabular}
\end{table}
Thus the treasure is in the garage.

\section*{Problem 5}
\begin{enumerate}
  \item True
  \item True
  \item False / Counterexample : $x=36$
  \item True
  \item False / Counterexample : $x=36$
\end{enumerate}

\section*{Problem 6}
Let's suppose $n$ is even when $(n^2+n+1)$ is even, for any integer $n$. 

\vspace{0.3cm} We can set $n=2k$ where $k \in \mathbb{Z}$. Then it follows that
\begin{align*}
  n^2 + n + 1 &= (2k)^2 + 2k + 1 \\
  &= 2 \times (2k^2 + k) + 1
\end{align*}
This shows that $(n^2+n+1)$ is odd, but according to the assumption $(n^2+n+1)$ should be even.

\vspace{0.3cm} Therefore, using `proofs by contradiction', we can say that $n$ should be odd.

\section*{Problem 7} 
Let's suppose that a real number $x$ which satisfies the equation is a rational number. 

\vspace{0.3cm} Solving the equation gives:
$$x = \frac{-b \pm \sqrt{b^2-4ac}}{2a}$$

For $x$ to be a rational number, $\sqrt{b^2-4ac}$ should be rational. Especially, since $a$, $b$, $c$ are integers, $(b^2-4ac)$ should be a perfect square. Thus we can set $b^2-4ac = k^2$, where $k \in \mathbb{N}$. This gives:
\begin{gather}
  b^2 - k^2 = 4ac \\
  (b+k)(b-k) = 4ac
\end{gather}
From (1), since $b^2$ is odd and $4ac$ is even, we can know that $k^2$ and $k$ should be odd. Then we can set $b=2b^*+1$, $k=2k^*+1$ and write (2) as:
\begin{equation}
  4(b^*+k^*+1)(b^*-k^*) = 4ac
\end{equation}
Now since $(b^*+k^*+1) + (b^*-k^*) = 2b^* + 1$ and is odd, either $(b^*+k^*+1)$ or $(b^*-k^*)$ is even. Then from (3), we can know that $4(b^*+k^*+1)(b^*-k^*)$ is a multiple of 8 and $ac$ should be even.

\vspace{0.3cm}
However, since this question is supposing $a$ and $c$ are odd, this is a contradiction. Therefore, $x$ is irrational.

\section*{Problem 8} 
Let's name each tile with an alphabet like the following. There are four types of tiles in total.
\begin{table}[H]
  \centering
  \renewcommand{\arraystretch}{1.2}
  \begin{tabular}{|c|c|c|c|c|c|c|c|c|c|}
    \hline
    a&b&c&d&a&b&c&d&a&b \\ \hline
    b&c&d&a&b&c&d&a&b&c \\ \hline
    c&d&a&b&c&d&a&b&c&d \\ \hline
    d&a&b&c&d&a&b&c&d&a \\ \hline
    a&b&c&d&a&b&c&d&a&b \\ \hline
    b&c&d&a&b&c&d&a&b&c \\ \hline
    c&d&a&b&c&d&a&b&c&d \\ \hline
    d&a&b&c&d&a&b&c&d&a \\ \hline
    a&b&c&d&a&b&c&d&a&b \\ \hline
    b&c&d&a&b&c&d&a&b&c \\ \hline
  \end{tabular}
\end{table}
Let's suppose we can tile the checkerboard with straight tetrominoes.

\vspace{0.3cm} A straight tetromino is a shape that has four tiles in a row. 
Therefore whererver we place a straight tetromino in the checkerboard, it will cover one of each tile. (`a', `b', `c', and `d')

\vspace{0.3cm} If we count the number of each tile, there are 25 `a' tiles, 26 `b' tiles, 25 `c' tiles, 24 `d' tiles.

\vspace{0.3cm} However, since we should fill the checkerboard only with straight tetrominoes, there should be an equal number of each tile.
This is a contradiction, and therefore we cannot tile the checkerboard.

\section*{Problem 9}
\setcounter{equation}{0}
\paragraph{(a)} False / Counterexample : $A = \qty{1}, B = \qty{2}, C = \qty{1, 2}$

\paragraph{(b)} True

\vspace{0.3cm} $A\cup C \subseteq B\cup C$ and $A\cap C \subseteq B\cap C$ are equivalent to the following.
\begin{align}
  A\cup C \subseteq B\cup C &\Leftrightarrow (A\cup C) \cap (B\cup C) = A \cup C \\
  A\cap C \subseteq B\cap C &\Leftrightarrow (A\cap C) \cap (B\cap C) = A \cap C
\end{align}
From (1), we can know
\begin{align}
  (A \cap B) \cup (A \cap C) &= \qty((A \cap B) \cup A) \cap \qty((A \cap B) \cup C) &(\rm Distributive \; Law) \nonumber \\
  &= A \cap \qty((A \cup C) \cap (B \cup C)) &(A \cap B \subseteq A \; / \; \rm Distributive \; Law) \nonumber \\
  &= A \cap (A \cup C) &(\rm From \; (1)) \nonumber \\
  &= A &(A \subseteq A \cup C)
\end{align}
Also (2) gives $(A\cap C) \cap (B\cap C) = A \cap B \cap C = B \cap (A \cap C) = A \cap C$, therefore 
\begin{equation}
  A \cap C \subseteq B
\end{equation}
From (3) and (4), we can know
\begin{align*}
  A \cap B &= \qty((A \cap B) \cup (A \cap C)) \cap B &(\rm From \; (3))\\
  &= \qty((A \cap B) \cap B) \cup \qty((A \cap C) \cap B) & (\rm Distributive \; Law) \\
  &= (A \cap B) \cup (A \cap C) &(A \cap B \subseteq B \; / \; \rm From \; (4)) \\
  &= A &(\rm From \; (3))
\end{align*}
Therefore $A \subseteq B$. 

\section*{Problem 10}
\paragraph{(a)} $f$ is surjective.

\vspace{0.3cm} The statement ``$f$ is surjective'' can be written as
$$\forall n \in \mathbb{Z}, \; \exists S \in F {\rm \; such \; that \;} f(S) = n$$
Choosing $S = \qty{n}$ gives $f(S) = n$. Therefore the proposition is true.

\paragraph{(b)} $f$ is not injective.

\vspace{0.3cm} The statement ``$f$ is injective'' can be written as
$$\forall S_1, S_2 \in F, \; {\rm if} \; S_1 \neq S_2 \; {\rm then} \; f(S_1) \neq f(S_2)$$

\vspace{0.3cm} We can find a counterexample: $f(\qty{0}) = f(\qty{1, -1}) = 0$

\section*{Problem 11}
\paragraph{(a)} Since $f$ is not injective, it is not a bijection.

\vspace{0.3cm} We can find a counterexample: $f(1, 1) = f(2, 2) = 0$


\paragraph{(b)}
$S = \qty{(n, 1) \,|\, n \in \mathbb{N}} \cup \qty{(1, n) \,|\, n \in \mathbb{N}} = \qty{(1,1), (2,1), (1,2), (3,1), (1,3), \cdots}$

\vspace{0.3cm} $f$ is a bijection over the given set $S$. We can prove this by showing $f$ is both surjective and injective.

\vspace{0.3cm} First, to show $f$ is surjective, we should prove ``$\,\forall k \in \mathbb{Z}, \; \exists \, T \in S {\rm \; such \; that \;} f(T) = k$''

\vspace{0.3cm} Choosing $T = \begin{cases} (1+k, 1)  & (k \geq 0) \\ (1, 1-k) & (k < 0) \end{cases}$ from $S$ gives $f(T) = k$. Therefore $f$ is surjective.

\vspace{0.3cm} Second, to show $f$ is injective, we should prove ``$\,\forall \,T_1, T_2 \in S, \; {\rm if} \; f(T_1) = f(T_2) \; {\rm then} \; T_1 = T_2$''.

\vspace{0.3cm} Let's say $S_1 = \qty{(n, 1) \,|\, n \in \mathbb{N}}$ and $S_2 = \qty{(1, n) \,|\, n \in \mathbb{N}, \; n > 1}$.
Then $S_1 \cup S_2 = S$ and $S_1 \cap S_2 = \emptyset$. 

\vspace{0.3cm} We can choose $T_1$ and $T_2$ from $S$ in four ways. 
\renewcommand\labelenumi{(\theenumi)}
\begin{enumerate}
  \item $T_1, T_2 \in S_1$ \\
  When $T_1 = (n_1, 1)$ and $T_2 = (n_2, 1)$, 
  if $f(T_1) = f(T_2)$ then $n_1 - 1 = n_2 - 1$ and $n_1 = n_2$. $\therefore T_1 = T_2$.
  \item $T_1, T_2 \in S_2$ \\
  When $T_1 = (1, n_1)$ and $T_2 = (1, n_2)$, 
  if $f(T_1) = f(T_2)$ then $1 - n_1 = 1 - n_2$ and $n_1 = n_2$. $\therefore T_1 = T_2$.
  \item $T_1 \in S_1$, $T_2 \in S_2$ \\
  When $T_1 = (n_1, 1)$ and $T_2 = (1, n_2)$, $f(T_1) = n_1 - 1 \geq 0$ and $f(T_2) = 1 - n_2 < 0$. Therefore always $f(T_1) \neq f(T_2)$, and the statement is always true. 
  \item $T_1 \in S_2$, $T_2 \in S_1$ \\
  Same as case (3), statement is true.
\end{enumerate}
Therefore $f$ is injective. Since $f$ is both surjective and injective, $f$ is a bijection.

\section*{Problem 12}
If $S$ is a finite set of $n$ elements ($n = 1, 2, 3, \cdots$), $\mathcal P(S)$ has $2^n$ elements. 

\vspace{0.3cm} When $S$ is an infinite set, let's suppose such a function $f$ exists. We can also define $T = \{s \in S | s  \notin f (s) \}$.

\vspace{0.3cm} Since $T$ consists of elements of $S$, it is a subset of $S$ and is an element of $\mathcal P(S)$. Let's think of an arbitary element $s \in S$. 
We can consider $s$ in two ways.
\begin{enumerate}
  \item $s \in T$ \\
  By the definition of $T$, $s \notin f(s)$ and therefore $T \neq f(s)$.
  \item $s \notin T$ \\
  By the definition of $T$, $s \in f(s)$ and therefore $T \neq f(s)$.
\end{enumerate}
Since $f$ is an onto function, for all $s$, $f(s)$ makes up $\mathcal P(S)$.
Also $T \neq f(s)$ for any $s$, so $T\notin \mathcal P(S)$. 
However, $T \in \mathcal P(S)$ according to its definition, so this is a contradiction. 
Therefore an onto function $f$ does not exist.

\vspace{0.3cm} $\therefore$ For both finite and infinite sets, an onto function $f$ from $S$ to $\mathcal P(S)$ does not exist.
\end{document}